\documentclass[12pt]{article}
\usepackage{fontspec}
\setmainfont{LXGW WenKai}[AutoFakeBold=2.5] %% 如果有兼容性问题,注释本行,BoldFont=黑体
\usepackage[fontset=none]{ctex} %% 中文支持自动换行,默认不设置字体
\usepackage{hyperref} %% 插入链接用

\title{Math Modeling Note}
\author{范某}
\date{}

\begin{document}
\maketitle

\section*{前置说明}

此处是一个LaTeX使用的介绍.

\subsection*{环境配置}

我这里非常想要使用自定义字体“霞鹜文楷”,所以使用了XeLaTeX作为
编译引擎,这样就可以非常简单地使用系统上安装的字体. 设置见settings.json.
但就害怕会有兼容性问题,不过目前发现的问题都能得到解决.

\leavevmode

我目前发现的问题有:
\begin{enumerate}
    \item 无法使用加粗 \textbf{加粗} 只能使用AutoFakeBold或者是BoldFont指定加粗字体;
    \item 无法自动换行;需要使用sloppypar环境,或者必须要有CTeX环境,但是使用了CTeX环境又会冲突....;为了解决这一问题,在调用CTeX时设置fontset=none即可
\end{enumerate}

\noindent \textbf{注}:XeLaTeX的配置参考自\href{https://tex.stackexchange.com/questions/226/installing-ttf-fonts-in-latex}{Installing TTF fonts in LaTeX},
以及\href{https://tex.stackexchange.com/questions/564758/how-to-use-visual-studio-code-latex-workshop-with-xelatex}{VSCode: how to use xelatex}

\leavevmode

\subsection*{为什么用VSCode进行\LaTeX 编辑}

目前大部分的需求都能被Markdown语法解决,并且转成PDF也非常简单. 但是在专业排版方面,Markdown不如\LaTeX.

此次尝试使用Visual Studio Code中的Live Share插件进行合作,作为Overleaf的替代.

然而最最最最重要的是,VSCode中可以进行\textbf{版本控制},只需要Ctrl + s即可保存一次版本,应该没有更简单的了.

\leavevmode

\subsection*{如何在Visual Studio中编译\LaTeX}

配置方面:首先需要安装TeX Live. Windows系统按照文档上说不需要设置环境变量,然后在VSCode中安装LaTeX插件即可使用,
并按照需要settings.json文件中设置格式,控制编译引擎和输出文件位置.

我这里设置的json.文件中指定将输出放在out子文件夹下.

处理报错:需要查看.log后缀的文件,以校对错误.

查看输出的PDF:因为我直接放到out文件夹中,似乎没有办法实时编译后查看.

重新编译:命令行中Kill LaTeX Compiler即可.

\subsection*{中文文档符号规范}

所有的句号写为英文句号“.”. 

其他保持为中文符号.

\break

emmmm

\end{document}